% !TeX spellcheck = en_US
\begin{abstract}
	Automatic translation, natural language processing and other language related tasks are among the most challenging problems in computational linguistics. The ability of a machine to understand written and spoken language led to notable improvements in tasks such as documents digitization, speech synthesis and text mining, but text recognition in handwritten documents is still one of the most difficult task to deal with. However, the further improvements on deep neural networks for object recognition have definitely amended the performance in this area of research. Convolutional Neural Networks (CNN) and Recurrent Neural Networks (RNN) in particular have been successfully applied to linguistic and vision tasks. Their ability to learn complex, non linear functions has been proven ideal to deal with the overwhelming number of features concealed in images and the complexity of languages. In this report, we show a deep-learning-based solution to the task proposed in the Kaggle competition “Kuzushiji Recognition”. The goal of the competition is translating ancient cursive handwritten Kuzushiji pages into modern Japanese. More precisely, a great number of different possible characters must be automatically detected and classified. Our approach is focused on a region-based model which uses a CenterNet like neural network to detect characters and then translates them into modern Japanese using a simple CNN. It resulted in 0.77 IoU score in detection, 0.94 accuracy of classification and a 0.797 modified $F_1$ score on the competition test set.
\end{abstract}


\begin{keywords}
	Kuzushiji - text recognition - object detection
\end{keywords}
