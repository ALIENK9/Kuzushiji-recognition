% !TeX spellcheck = en_US
\begin{abstract}
	Automatic translation, natural language processing and other language related tasks are among the most challenging problems in computational linguistics. The ability of a machine to understand written and spoken language led to notable improvements in tasks such as documents digitization, speech synthesis and text mining, but text recognition in handwritten historical documents is still one of the most difficult task to deal with. However, the further improvements on deep neural networks for object recognition have definitely amended the performance in this area of research. CNN and RCNN in particular have been successfully applied to linguistic and vision tasks. Their ability to learn complex, non linear functions and the tendency towards layer specialization have been proven ideal to deal with the overwhelming number of features concealed in images and the complexity of languages. In this report, we show a deep-learning-based solution to the task proposed in the Kaggle competition “Kuzushiji Recognition”. The goal of the competition is translating ancient handwritten Kuzushiji pages into modern Japanese. More precisely, a great number of different possible characters must be automatically detected and classified. Our approach is focused on a deep neural network called CenterNet to detect characters, which are then translated into modern Japanese using a simple CNN. It resulted in [DETECTION ACCURACY] accuracy of detection and [CLASSIFICATION ACCURACY] accuracy of classification, approximately in line with the state-of-the-art.
\end{abstract}


\begin{keywords}
	kuzushiji - text recognition - object detection
\end{keywords}
