% !TeX spellcheck = en_US
\section{CONCLUSIONS}
\label{sec:conclusions}

The objective of the project was accurately recognizing handwritten cursive Kuzushiji characters in ancient historical documents, which has been accomplished via a region-based object recognition approach. The illustrated model featured a CenterNet like detector followed by a CNN classifier, and achieved satisfying performances.\\

We concluded the competition with a sensible overall Kaggle score of \textcolor{red}{SCORE}. However, many improvements could have been tested. For example, a more tailored tiling and preprocessing would have probably resulted in a better accuracy of the predictions. Furthermore it could have been worth to try different encoding and decoding architectures for the detection model, like VGG and Hourglass, which are also referenced in the CenterNet paper. Using a totally different approach, Mask R-CNN could have been used to detect the characters, since it can generate a pixel-wise mask that seems suitable to recognize characters shapes. The usage of a recurrent model to add context information on each character classification, could have resulted in an improvement as well. Like reported on the dataset description of the competition, indeed, some Kuzushiji characters are very similar, differing just for small details, and their meaning can be more easily inferred from context than by looking at the single character itself.
